\documentclass[ignorenonframetext,]{beamer}
\setbeamertemplate{caption}[numbered]
\setbeamertemplate{caption label separator}{: }
\setbeamercolor{caption name}{fg=normal text.fg}
\beamertemplatenavigationsymbolsempty
\usepackage{lmodern}
\usepackage{amssymb,amsmath}
\usepackage{ifxetex,ifluatex}
\usepackage{fixltx2e} % provides \textsubscript
\ifnum 0\ifxetex 1\fi\ifluatex 1\fi=0 % if pdftex
  \usepackage[T1]{fontenc}
  \usepackage[utf8]{inputenc}
\else % if luatex or xelatex
  \ifxetex
    \usepackage{mathspec}
  \else
    \usepackage{fontspec}
  \fi
  \defaultfontfeatures{Ligatures=TeX,Scale=MatchLowercase}
\fi
% use upquote if available, for straight quotes in verbatim environments
\IfFileExists{upquote.sty}{\usepackage{upquote}}{}
% use microtype if available
\IfFileExists{microtype.sty}{%
\usepackage{microtype}
\UseMicrotypeSet[protrusion]{basicmath} % disable protrusion for tt fonts
}{}
\newif\ifbibliography
\hypersetup{
            pdftitle={Group 1: Tennis},
            pdfauthor={Olivia Beck, Emma Wilson, Ryan Volkert},
            pdfborder={0 0 0},
            breaklinks=true}
\urlstyle{same}  % don't use monospace font for urls
\usepackage{graphicx,grffile}
\makeatletter
\def\maxwidth{\ifdim\Gin@nat@width>\linewidth\linewidth\else\Gin@nat@width\fi}
\def\maxheight{\ifdim\Gin@nat@height>\textheight0.8\textheight\else\Gin@nat@height\fi}
\makeatother
% Scale images if necessary, so that they will not overflow the page
% margins by default, and it is still possible to overwrite the defaults
% using explicit options in \includegraphics[width, height, ...]{}
\setkeys{Gin}{width=\maxwidth,height=\maxheight,keepaspectratio}

% Prevent slide breaks in the middle of a paragraph:
\widowpenalties 1 10000
\raggedbottom

\AtBeginPart{
  \let\insertpartnumber\relax
  \let\partname\relax
  \frame{\partpage}
}
\AtBeginSection{
  \ifbibliography
  \else
    \let\insertsectionnumber\relax
    \let\sectionname\relax
    \frame{\sectionpage}
  \fi
}
\AtBeginSubsection{
  \let\insertsubsectionnumber\relax
  \let\subsectionname\relax
  \frame{\subsectionpage}
}

\setlength{\parindent}{0pt}
\setlength{\parskip}{6pt plus 2pt minus 1pt}
\setlength{\emergencystretch}{3em}  % prevent overfull lines
\providecommand{\tightlist}{%
  \setlength{\itemsep}{0pt}\setlength{\parskip}{0pt}}
\setcounter{secnumdepth}{0}

\title{Group 1: Tennis}
\author{Olivia Beck, Emma Wilson, Ryan Volkert}
\date{December 12, 2019}

\begin{document}
\frame{\titlepage}

\begin{frame}{The Game of Tennis}

\begin{itemize}
\item
  Point: Smallest unit of measurement (Love-15-30-40-game)
\item
  Game: A game is one when a player reaches 4 points with at least a 2
  point advantage
\item
  Set: A set consists of 6 games and is won by the player who reaches 6
  games first
\item
  Advantage Set: If a game score of 6-6 is reached and advantage set
  rules are used, a player can only win a set with a 2 game lead
\item
  Matches: Best of 3 set (for women) or 5 sets (for men)
\item
  Tie-break game: If a game score of 6-6 is reached and tie-break set
  rules are used. In a tie-break game, a player/team must reach 7 points
  with a two point advantage to win
\end{itemize}

\end{frame}

\begin{frame}{Values}

Things we need to know:

\begin{itemize}
\item
  Empirical probability of winning a rally on serve
\item
  Empirical probability of winning a game on serve \pause
\end{itemize}

Things we need to calculate:

\begin{itemize}
\item
  Theoretical probability of winning a game on serve
\item
  Theoretical probability of winning a set
\item
  Theoretical probability of winning a match
\item
  Theoretical probability of winning the tie-breaker
\end{itemize}

\end{frame}

\begin{frame}{Slide with R Output}

\begin{table}[H]
\centering
\resizebox{\linewidth}{!}{
\begin{tabular}{rrrr}
\toprule
P(Win a Rally) & Empirical P(Win a Game) & Paper P(Win a Game) & Our P(Win a Game)\\
\midrule
0.69 & 0.71 & 0.89 & 0.89\\
0.63 & 0.80 & 0.79 & 0.79\\
0.65 & 0.85 & 0.83 & 0.83\\
0.63 & 0.77 & 0.79 & 0.79\\
\bottomrule
\end{tabular}}
\end{table}

\end{frame}

\begin{frame}{Slide with Plot}

\includegraphics{Group1Prez_files/figure-beamer/pressure-1.pdf}

\end{frame}

\end{document}
